\documentclass[12pt]{article}
\usepackage{fullpage,hyperref,amsmath,amsfonts,color,microtype}
\bibliographystyle{amsplain}

\newcommand{\p}{\partial}
\newcommand{\mgphi}{|\nabla \phi|}
\newcommand{\GPa}{\textrm{~GPa}}
\newcommand{\MPa}{\textrm{~MPa}}
\newcommand{\um}{\textrm{~$\mu$m}}
\newcommand{\drt}[1]{\frac{d #1}{d t}}
\newcommand{\prx}[1]{\frac{\p #1}{\p x}}
\newcommand{\pry}[1]{\frac{\p #1}{\p y}}
\newcommand{\Dpl}{D^\textrm{pl}}
\newcommand{\tT}{\tilde{t}}
\newcommand{\xT}{\tilde{x}}
\newcommand{\yT}{\tilde{y}}
\newcommand{\rT}{\tilde{r}}
\newcommand{\uT}{\tilde{u}}
\newcommand{\vT}{\tilde{v}}
\newcommand{\KT}{\tilde{K}}
\newcommand{\pT}{\tilde{p}}
\newcommand{\sT}{\tilde{s}}
\newcommand{\qT}{\tilde{q}}
\newcommand{\tauT}{\tilde{\tau}}
\newcommand{\muT}{\tilde{\mu}}
\newcommand{\drtT}[1]{\frac{d #1}{d \tT}}
\newcommand{\prxT}[1]{\frac{\p #1}{\p \xT}}
\newcommand{\pryT}[1]{\frac{\p #1}{\p \yT}}
\newcommand{\QT}{\tilde{Q}}
\newcommand{\sY}{s_Y}
\newcommand{\Kfac}{K_I^*}
\newcommand{\sC}{\mathcal{C}}
\newcommand{\bs}{\bar{s}}
\newcommand{\scT}{\mathcal{T}}
\newcommand{\mub}{\bar{\mu}}
\newcommand{\TK}{\textrm{~K}}
\renewcommand{\vec}[1]{\mathbf{#1}}
\newcommand{\vus}{\vec{u}_*}

\definecolor{Blue}{rgb}{0.12,0.05,0.9}
\newcommand{\q}[1]{\textcolor{Blue}{#1}}

\DeclareMathOperator{\arcsinh}{arcsinh}
\DeclareMathOperator{\arctanh}{arctanh}

\begin{document}
\section*{Simulations of a crack tip using the thermal STZ model}
We aim to make use of the thermal version of the STZ model described by
Langer~\cite{langer08} to investigate crack formation. The parameters in the
model are chosen to closely match Vitreloy 1
($\textrm{Zr}_{41}\textrm{Ti}_{14}\textrm{Cu}_{12.5}\textrm{Ni}_{10}\textrm{Be}_{22.5}$).

\subsection*{Governing equations and scaling}
We consider a plane strain formulation in the $x$ and $y$ coordinates, but make
use of a full three-dimensional stress tensor given by
\[
\sigma = \left(
\begin{array}{ccc}
  -p + s -q & \tau & 0 \\
  \tau & -p-s-q & 0 \\
  0 & 0 & -p + 2q \\
\end{array}
\right).
\]
Here, $p$ is the pressure, $s$ and $\tau$ are the components of deviatoric
stress within the $xy$ plane, and $q$ is the component of deviatoric stress in
the $z$ direction out of the plane. The magnitude of the deviatoric stress
tensor is $\bar{s}=\sqrt{s^2+\tau^2+3q^2}$. We make use of a system of linear
elastoplasticity given by
\begin{eqnarray}
  \label{eq:sys_start} \rho_0 \drt{u}&=&-\prx{p}-\prx{q}+\prx{s}+\pry{\tau} \\
  \rho_0 \drt{v}&=&-\pry{p}-\pry{q}-\pry{s}+\prx{\tau} \\
  \drt{p} &=& -K \left(\prx{u} +\pry{v}\right) \\
  \drt{q} &=& -\frac{\mu}{3} \left(\prx{u} + \pry{v}\right) -\frac{2\mu q \Dpl}{\bar{s}}\\
  \drt{s} &=& 2\omega \tau + \mu\left( \prx{u} - \pry{v} \right) - \frac{2\mu s \Dpl}{\bar{s}} \\
  \label{eq:sys_end} \drt{\tau} &=& - 2\omega s + \mu \left( \pry{u} + \prx{v} \right) - \frac{2\mu \tau \Dpl}{\bar{s}}
\end{eqnarray}
where $u$ and $v$ are the horizontal and vertical components of velocity,
and the angular velocity is $\omega= (\p v /\p x - \p u / \p y)/2$. $K$ and
$\mu$ are the elastic constants and $\rho_0$ is the density of the material.
The simulation makes use of rescaled dimensionless units. The stress parameters
are rescaled in terms of the yield stress $\sY$, so that
\[
\pT=\frac{p}{\sY}, \qquad \qT = \frac{q}{\sY}, \qquad \sT = \frac{s}{\sY}, \qquad \tauT = \frac{\tau}{\sY}, \qquad \KT = \frac{K}{\sY}, \qquad \muT = \frac{\mu}{\sY}.
\]
Lengths are scaled according to the initial radius of curvature of the notch,
$\varrho$. The simulation uses $-M<\tilde{x}<M$, $-M<\tilde{y}<M$ in the
rescaled coordinates for some constant $M$. The transverse wave speed is
\[
c_s = \sqrt{\frac{\mu}{\rho_0}}.
\]
The rescaled time coordinate $\tT$ is then related to real time $t$ according
to
\[
t = \frac{\varrho\tT}{c_s}. 
\]
The simulations make use of dimensionless velocities $\uT = u/c_s, \vT =
v/c_s$. Using these scalings, Eqs.~\ref{eq:sys_start} to \ref{eq:sys_end} can
be rewritten as
\begin{eqnarray}
  \muT \drtT{\uT}&=&-\prxT{\pT}-\prxT{\qT}+\prxT{\sT}+\pryT{\tauT} \\
  \muT \drtT{\vT}&=&-\pryT{\pT}-\pryT{\qT}-\pryT{\sT}+\prxT{\tauT} \\
  \drtT{\pT} &=& -\KT \left(\prxT{\uT} +\pryT{\vT}\right) \\
  \drtT{\qT} &=& -\frac{\muT}{3} \left(\prxT{\uT} +\pryT{\vT}\right) - \mu \left(\frac{2\varrho \Dpl}{c_s}\right) \frac{\qT}{\tilde{\bs}} \\
  \label{eq:res_s} \drtT{\sT} &=& 2\tilde{\omega} \tauT + \muT\left( \prxT{\uT} - \pryT{\vT} \right) - \muT \left(\frac{2 \varrho \Dpl}{c_s}\right) \frac{\sT}{\tilde{\bs}} \\
  \label{eq:res_tau} \drtT{\tauT} &=& - 2\tilde{\omega} \sT + \muT \left( \pryT{\uT} + \prxT{\vT} \right) - \muT \left(\frac{2 \varrho \Dpl}{c_s}\right) \frac{\tauT}{\tilde{\bs}}.
\end{eqnarray}

\subsection*{Elastic constants}
The material has a Young's modulus of $E=101\GPa$ and a Poisson ratio
of $\nu=0.35$~\cite{wang04} which gives a bulk modulus of
\[
K=\frac{101\GPa}{3(1-2\times0.35)} = 114.1 \GPa
\]
and a shear modulus of
\[
\mu=\frac{101\GPa}{2(1+0.35)} = 37.4 \GPa.
\]
An appropriate density is $\rho_0=6.125\times10^3\textrm{ kg m}^{-3}$ and thus
the transverse wave speed is given by
\begin{eqnarray*}
c^2_s &=& \frac{37.4\GPa}{6.125 \textrm{ kg m}^{-3}} \\
c_s &=& 2.471 \times 10^3 \textrm{ m s}^{-1}.
\end{eqnarray*}
We make use of a yield stress of $\sY \approx s_0 = 1.1 \GPa$, and thus the
rescaled simulation paramaters are
\begin{eqnarray*}
  \tilde{K}&=&\frac{K}{\sY} = 103.7 \\
  \tilde{\mu}&=&\frac{\mu}{\sY} = 34.0.
\end{eqnarray*}


\subsection*{Plastic deformation using the STZ model}
The amount of plastic deformation is given by 
\begin{equation}
  \label{eq:strainrate}
  \Dpl = \frac{\epsilon_0}{\tau_0} \sC(s) e^{-1/\chi} [\scT(\bs) - M(\bs)]
\end{equation}
where
\[
M(\bs) = \frac{s_0}{2\bs} \left[1+\frac{\bs}{s_0}\scT(\bs) + \frac{\rho(T)}{2\sC(\bs)}\right]
- \frac{s_0}{2\bs} \sqrt{\left[1+\frac{\bs}{s_0} \scT(\bs) + \frac{\rho(T)}{2\sC(\bs)}\right]^2 - 4 \frac{\bs}{s_0} \scT(\bs)}.
\]
To use this, we must compute $\scT(\bs)$, $\sC(\bs)$, and $\rho(T)$, by making
use of the functional forms suggested by Langer:
\begin{eqnarray*}
  \scT(\bs) &=& \tanh \left[\frac{T_E}{T} \sinh (\bs/\mub) \right] \\
  \sC(\bs) &=& \exp \left[ - \frac{T_E}{T} \cosh (\bs/\mub) \right] \cosh\left[\frac{T_E}{T} \sinh(\bs/\mub) \right] \left[ 1+\left(\frac{\bs}{s_1}\right) \right]^{n/2} \\
  \rho(T) &=& \left\{
  \begin{array}{ll}
    e^{-\alpha(T)} & \qquad \textrm{for $T>T_0$} \\
    0 & \qquad \textrm{for $T<T_0$}
  \end{array}
  \right. \\
  \alpha(T) &=& \frac{T_1}{T-T_0} e^{-a(T-T_0)/(T_A - T_0)}.
\end{eqnarray*}
By comparison with Eqs.~\ref{eq:res_s} and \ref{eq:res_tau}, we see that
many of the STZ parameters can be scaled out by introducing
\begin{equation}
  \label{eq:nu_intro}
  \nu = \frac{2 \varrho \epsilon_0}{c_s \tau_0}.
\end{equation}
We also make use of the following equation of motion for the effective
temperature:
\begin{equation}
  \label{eq:chi}
  \tau_0 \tilde{c}_0 \frac{d\chi}{dt} = e^{-1/\chi} \Gamma(\bs) \chi \left[ 1 - \frac{\chi}{\chi_0} \right] + \kappa e^{-\beta/\chi} \rho(T) \chi \left[ 1 - \frac{T_Z\chi}{T}\right].
\end{equation}
To use this, we need the following additional relations:
\begin{eqnarray*}
\Gamma(\bs)&=& \frac{2 \sC[ \scT - M][(\bs/s_0) - \xi(M)] + M\xi(M) \rho(T)}{1-M\xi(M)} \\
\xi(m) &=& \frac{1}{s_0} \scT^{-1}(m) = \frac{\mub}{\q{s_0}} \arcsinh\left[\frac{T}{T_E} \arctanh(m)\right]. \\
\end{eqnarray*}
The term shown in blue does not feature in Ref.~\cite{langer08} but appears
necessary for dimensional consistency. If Eq.~\ref{eq:chi} is written in terms
of the coordinate $\tilde{t}=c_s t / \varrho$, and we make use of
$\tilde{c}_0=c_0/\epsilon_0\nu_z$, then
\[
\frac{d\chi}{d\tilde{t}} = \frac{\varrho \epsilon_0 \nu_z}{\tau_0 c_0 c_s}
\left[e^{-1/\chi} \Gamma(\bs) \chi \left[ 1 - \frac{\chi}{\chi_0} \right] +
\kappa e^{-\beta/\chi} \rho(T) \chi \left[ 1 - \frac{T_Z\chi}{T}\right]\right].
\]
Using Eq.~\ref{eq:nu_intro}, this can be written as
\[
\frac{d\chi}{d\tilde{t}} = \frac{\nu\nu_z}{2 c_0}
\left[e^{-1/\chi} \Gamma(\bs) \chi \left[ 1 - \frac{\chi}{\chi_0} \right] +
\kappa e^{-\beta/\chi} \rho(T) \chi \left[ 1 - \frac{T_Z\chi}{T}\right]\right].
\]
Currently, we make use of $\nu=10^9$, which is higher than previous simulation,
but reflective of the fact the the effective temperatures considered are
smaller, leading to a considerably smaller $\exp(-1/\chi)$ factor. We set
$\nu_z=200$. There are a large number of other free parameters in this model.
Within the simulation, the temperatures are stored using their values in
Kelvin, making use of
\[
T_0 = 250 \TK, \quad T_A = 1000 \TK, \quad T_E = 3000\TK, \quad T_Z= 16000\TK, \quad T_1 = 31000\TK.
\]
The stress parameters are
\[
\mub=0.1\GPa, \quad s_0= 1.1\GPa, \quad s_1=1.0\GPa,
\]
and the remaining parameters are
\[
n=1, \quad a=3, \quad \beta =1, \quad \chi_0=0.06, \quad \kappa=0.5.
\]

\subsection*{Boundary conditions}
An appropriate value for the fracture toughness is $\Kfac = 50 \MPa
\sqrt{\textrm{m}}$. We use the Irwin crack tip solutions where the $K_I$ is
slowly increased. We make use of
\begin{equation}
  \label{eq:rate}
  K_I(t) = \Kfac \left(\lambda+\frac{\gamma c_s t}{\varrho}\right).
\end{equation}
where $\lambda$ and $\gamma$ are dimensionless. The boundary conditions for
velocity from the Irwin crack tip solutions are
\begin{eqnarray*}
u(r,\theta,t) &=& Q(r,t) (1+\nu) [(2\kappa -1)\cos(\theta/2) - \cos(3\theta/2)] \\
v(r,\theta,t) &=& Q(r,t) (1+\nu) [(2\kappa +1)\sin(\theta/2) - \sin(3\theta/2)]
\end{eqnarray*}
where
\[
Q(r,t) = \frac{1}{2E} \sqrt{\frac{r}{2\pi}} \frac{d}{dt} K_I(t)
= \frac{\Kfac\gamma c_s}{2E\varrho} \sqrt{\frac{r}{2\pi}}.
\]
It is appropriate to introduce $\QT=Q/c_s$, and hence
\begin{eqnarray*}
  \QT(\rT,\tT) &=& \frac{\gamma}{\varrho} \frac{50 \MPa \sqrt{\textrm{m}}}{101 \GPa} \sqrt{\frac{\varrho\rT}{2\pi}} \\
  &=& 4.95 \times 10^{-4} \, \gamma \sqrt{\frac{\rT}{2\pi}} \sqrt{\textrm{m}/\varrho}
\end{eqnarray*}
The cracking process can be investigated for different values of $\lambda$ and
$\gamma$. From Eq.~\ref{eq:rate}, it can be seen that a value of
$\gamma=1$ corresponds to a very large strain rate. Although we generally
consider $\lambda=0$ initially, we can impose a stress field taken from the
Irwin crack tip solutions in other cases. For $t=0$, we have
\begin{eqnarray*}
  \sigma_{xx}(r,\theta) &=& W(r) \cos(\theta/2)[1-\sin(\theta/2)\sin(3\theta/2)] \\
  \sigma_{xy}(r,\theta) &=& W(r) \cos(\theta/2)[1+\sin(\theta/2)\sin(3\theta/2)] \\
  \sigma_{yy}(r,\theta) &=& W(r) \sin(\theta/2) \cos(\theta/2) \cos(3\theta/2)
\end{eqnarray*}
where
\[
W(r) = \frac{K_I(0)}{\sqrt{2\pi r}} = \frac{\lambda\Kfac}{\sqrt{2\pi r}}.
\]
Introducing $\tilde{W}=W/\sY$, we obtain
\[
\tilde{W}(r) = 4.5\times 10^{-2} \, \frac{\lambda}{\sqrt{2\pi \tilde{r}}} \sqrt{ \textrm{m}/\varrho}
\]
This can be used to initialize the rescaled stresses $\tilde{p}$, $\tilde{s}$, and
$\tilde{\tau}$ in simulation.

So far, $\gamma$ has been set to $0.01$. Fracture occurs in the range $100<\tilde{t}<200$ which appears to be in approximate agreement with the fracture toughness.

\subsection*{Void nucleation}
Currently, voids are nucleated by scanning the pressure field at each timestep.
If a gridpoint position $\vec{x}_v$ is found where the rescaled pressure
$\tilde{p}(\vec{x}_v)$ is lower than a critical value $\tilde{p}_\textrm{cr}$,
then a void is nucleated by constructing a new level set function
$\phi_\textrm{new}(\vec{x})$ in terms of the old one $\phi_\textrm{old}$
according to
\[
\phi_\textrm{new}(\vec{x}) = \min\{ \phi_\textrm{old}(\vec{x}), |\vec{x}-\vec{x}_v|-r_v\}.
\]
Here, $r_v$ is the size of the void, and typically, a value of 0.9 times the
grid spacing is used. So far, values of $\tilde{p}_\textrm{cr}$ of -4.5 and -5.0
have been used.

\subsection*{Quasistatic limit}
To mimic physically realistic strain rates, the dimensionless velocities $\uT$
and $\vT$ must be small. One possible way to consider this would be with
\begin{eqnarray*}
\uT &=& \epsilon U \\
\vT &=& \epsilon V \\
\tT &=& T/\epsilon,
\end{eqnarray*}
and aiming that $U$, $V$, and $T$ become $O(1)$ quantities. The system of
equations would then become 
\begin{eqnarray}
  \label{eq:eps_u} \muT \epsilon^2 \frac{dU}{dT}&=&-\prxT{\pT}-\prxT{\qT}+\prxT{\sT}+\pryT{\tauT} \\
  \label{eq:eps_v} \muT \epsilon^2 \frac{dV}{dT}&=&-\pryT{\pT}-\pryT{\qT}-\pryT{\sT}+\prxT{\tauT} \\
  \frac{d\pT}{dT} &=& -\KT \left(\prxT{U} +\pryT{V}\right) \\
  \frac{d\qT}{dT} &=& - \frac{\muT}{3} \left(\prxT{U} +\pryT{V}\right) - \frac{\muT}{\epsilon} \left(\frac{2 \varrho \Dpl}{c_s}\right) \frac{\qT}{\bs} \\
  \frac{d\sT}{dT} &=& 2\Omega \tauT + \muT \left( \prxT{U} - \pryT{V} \right) - \frac{\muT}{\epsilon} \left(\frac{2 \varrho \Dpl}{c_s}\right) \frac{\sT}{\bs} \\
  \label{eq:eps_tau} \frac{d\tauT}{dT} &=& - 2\Omega \sT + \muT \left( \pryT{U} + \prxT{V} \right) - \frac{\muT}{\epsilon} \left(\frac{2 \varrho \Dpl}{c_s}\right) \frac{\tauT}{\bs}.
\end{eqnarray}
where $\Omega= (\p V /\p x - \p U / \p y)/2$. The $\epsilon^{-1}$ factor on the
$\Dpl$ term looks promising, since this can balance out the fact that this term
is small. The $\epsilon^2$ terms in Eqs.~\ref{eq:eps_u} and~\ref{eq:eps_v}
suggest that we may just want to solve
\begin{eqnarray}
  \label{eq:eps_u2} 0&=&-\prxT{\pT}-\prxT{\qT}+\prxT{\sT}+\pryT{\tauT} \\
  \label{eq:eps_v2} 0&=&-\pryT{\pT}-\pryT{\qT}-\pryT{\sT}+\prxT{\tauT},
\end{eqnarray}
or perhaps, if we include a viscosity,
\begin{eqnarray}
  \label{eq:eps_u3} -\kappa \nabla^2 U &=&-\prxT{\pT}-\prxT{\qT}+\prxT{\sT}+\pryT{\tauT}\\
  \label{eq:eps_v3} -\kappa \nabla^2 V &=&-\pryT{\pT}-\pryT{\qT}-\pryT{\sT}+\prxT{\tauT}.
\end{eqnarray}
With these equations, we have lost the ability to evolve $U$ and $V$
explicitly. One possible method of solution is to use a predictor--corrector
method, similar to the projection method for the incompressible Navier--Stokes
equations~\cite{chorin68}. Suppose that we first do compute intermediate
stresses~$\pT_*,\sT_*,\tauT_*$ by neglecting the strain rate terms, so that
\begin{eqnarray}
  \frac{\pT_*-\pT_n}{\Delta T} &=& -\vec{U}_n\cdot\nabla \pT_n \\
  \frac{\qT_*-\qT_n}{\Delta T} &=& -\vec{U}_n\cdot\nabla \qT_n - \frac{\muT}{\epsilon} \left(\frac{2 \varrho \Dpl_n}{c_s}\right) \frac{\qT_n}{\bs_n}\\  
  \frac{\sT_*-\sT_n}{\Delta T} &=& -\vec{U}_n\cdot\nabla \sT_n + 2\Omega_n \tauT_n - \frac{\muT}{\epsilon} \left(\frac{2 \varrho \Dpl_n}{c_s}\right) \frac{\sT_n}{\bs_n} \\
  \frac{\tauT_*-\tauT_n}{\Delta T} &=& -\vec{U}_n\cdot\nabla \tauT_n - 2\Omega_n \sT_n - \frac{\muT}{\epsilon} \left(\frac{2 \varrho \Dpl_n}{c_s}\right) \frac{\tauT_n}{\bs_n}.
\end{eqnarray}
We can then write
\begin{eqnarray}
  \label{eq:proj_start} \frac{\pT_{n+1}-\pT_*}{\Delta T} &=& - \KT \left(\prxT{U} +\pryT{V}\right) \\
  \frac{\qT_{n+1}-\qT_*}{\Delta T} &=& - \frac{\muT}{3} \left(\prxT{U} +\pryT{V}\right) \\
  \frac{\sT_{n+1}-\sT_*}{\Delta T} &=& \muT \left( \prxT{U} - \pryT{V} \right)  \\
  \label{eq:proj_end} \frac{\tauT_{n+1}-\tauT_*}{\Delta T} &=& \muT \left( \pryT{U} + \prxT{V} \right).
\end{eqnarray}
We now consider Eqs.~\ref{eq:eps_u3}~\&~\ref{eq:eps_v3}. We wish to enforce these
constraints for the $(n+1)$th timestep. From Equation~\ref{eq:eps_u3},
\begin{eqnarray*}
  -\kappa \nabla^2 U_{n+1} &=& -\prxT{\pT_{n+1}}-\prxT{\qT_{n+1}}+\prxT{\sT_{n+1}}+\pryT{\tauT_{n+1}}\\
  &=& -\prxT{\pT_*}-\prxT{\qT_*}+\prxT{\sT_*}+\pryT{\tauT_*} \\
  &&+\Delta T \left( (\muT+\KT') \frac{\p^2 U_{n+1}}{\p x^2} + \muT \frac{\p^2 U_{n+1}}{\p y^2} + \KT' \frac{\p^2 V_{n+1}}{\p x \p y} \right)
\end{eqnarray*}
and from Equation~\ref{eq:eps_v3},
\begin{eqnarray*}
  -\kappa \nabla^2 V_{n+1} &=& -\pryT{\pT_{n+1}}-\pryT{\qT_{n+1}}-\pryT{\sT_{n+1}}+\prxT{\tauT_{n+1}}\\
  &=& -\pryT{\pT_*}-\pryT{\qT_*}-\pryT{\sT_*}+\prxT{\tauT_*} \\
  && +\Delta T \left( \muT \frac{\p^2 V_{n+1}}{\p x^2} + (\muT+\KT') \frac{\p^2 V_{n+1}}{\p y^2} + \KT' \frac{\p^2 U_{n+1}}{\p x \p y} \right).
\end{eqnarray*}
where $\KT'=\KT+\frac{\muT}{3}$. Hence we must solve the system of equations
{\small\begin{eqnarray}
  \label{eq:double_multi1} (\muT+\KT'+\kappa') \frac{\p^2 U_{n+1}}{\p x^2} + (\muT+\kappa') \frac{\p^2 U_{n+1}}{\p y^2} + \KT' \frac{\p^2 V_{n+1}}{\p x \p y} &=& \frac{1}{\Delta T} \left(\prxT{\pT_*}+\prxT{\qT_*}-\prxT{\sT_*}-\pryT{\tauT_*}\right) \\
  \label{eq:double_multi2} (\muT+\kappa') \frac{\p^2 V_{n+1}}{\p x^2} + (\muT+\KT'+\kappa') \frac{\p^2 V_{n+1}}{\p y^2} + \KT' \frac{\p^2 U_{n+1}}{\p x \p y}&=& \frac{1}{\Delta T} \left( \pryT{\pT_*}+\pryT{\qT_*}+\pryT{\sT_*}-\prxT{\tauT_*} \right)
\end{eqnarray}}
where $\kappa'=\kappa/\Delta t$. This vector system of equations appears to be
solvable by the usual iterative methods like Jacobi, Gauss--Seidel, and
multigrid. Once this is solved, then we can use
Eqs.~\ref{eq:proj_start}--\ref{eq:proj_end} to compute $\pT_{n+1}, \sT_{n+1},
\tauT_{n+1}$.

\subsection*{Quasistatic boundary conditions}
To solve the coupled elliptic problem given in Eqs.~\ref{eq:double_multi1} \&
\ref{eq:double_multi2}, boundary conditions for $U$ and $V$ must be applied. At
the edges of the simulation region, we can make use of the Irwin crack tip
solutions. However, at the edge of the crack tip, we need to find an analog of
$\sigma \cdot \hat{\vec{n}}=\vec{0}$ that was used in the direct solve.

To make progress, it is helpful to consider the projection method for the
incompressible Navier--Stokes equations, where appropriate boundary conditions
for $p$ must be found. In this method, one first finds an intermediate velocity
$\vus$ according to
\begin{equation}
  \frac{\vus-\vec{u}_n}{\Delta t} = - (\vec{u}_n \cdot \nabla) \vec{u}_n + \nu \nabla^2 \vec{u}_n.
\end{equation}
The intermediate velocity may not satisfy the incompressibility constraint.
The velocity at the next timestep is given by
\begin{equation}
  \label{eq:projm}
  \vec{u}_{n+1}= \vus - \frac{\Delta t}{\rho} \nabla p_{n+1}.
\end{equation}
To make sure the velocity at $(n+1)$th timestep satisfies the incompressibility
constraint, we must solve the elliptic equation
\begin{equation}
  \nabla^2 p_{n+1} = \frac{\rho}{\Delta t} \nabla \cdot \vus.
\end{equation}
To use this, appropriate boundary conditions for $p$ must be found. Near a
wall, where one wishes to enforce $\vec{u}\cdot \hat{\vec{n}}=0$, a possible
strategy is to first solve for the intermediate $\vus$ using a free boundary
condition. Then, to enforce the boundary condition at the $(n+1)$th timestep,
Eq.~\ref{eq:projm} implies that
\begin{equation}
\frac{\Delta t}{\rho} \hat{\vec{n}} \cdot (\nabla p_{n+1}) = \hat{\vec{n}} \cdot \vus.
\end{equation}
There appears to be an analog to the quasistatic elastoplastic system. Rather
than enforce the stress boundary conditions during the computation of
$\sigma^*$, we can use free boundary conditions instead. Then, when
computing $U$ and $V$, enforcing $\sigma_{n+1} \cdot \hat{\vec{n}}=\vec{0}$
gives boundary conditions via Eqs.~\ref{eq:proj_start}--\ref{eq:proj_end}.
Specifically,
\[
  \frac{1}{\Delta t} \hat{\vec{n}} \cdot
\left(
\begin{array}{cc}
  -\pT_*-\qT_* + \sT_* & \tauT_* \\
  \tauT_* & -\pT_*-\qT_* -\sT_* \\
\end{array}
\right)
\]
\begin{equation}
  =\hat{\vec{n}} \cdot
\left(
\begin{array}{cc}
  -\KT'(U_x+V_y) - \muT(U_x-V_y) & -\muT(U_y+V_x) \\
  -\muT(U_y+V_x) & -\KT'(U_x+V_y) +\muT(U_x-V_y) \\
\end{array}
\right).
\end{equation}
This will give two equations, which is precisely the number of boundary
conditions required.

Numerically, this appears feasible, although the implementation of the above
conditions in the presence of a spatially varying $\hat{\vec{n}}$ may be
tricky. It does appear to have some numerical advantages over the previous
method. Up until now, during each iteration, the fields have been extrapolated
multiple times: they are extrapolated before the explicit step for stress, and
then extrapolated again before the elliptic solve. In the above method, using
the free boundary conditions, it appears to be very natural to extrapolate only
once at the very start of the iteration.

Since the boundary conditions above involve derivatives, they could pose
problems for free bodies, since there would be no way to set an overall
constant. However, this would appear to be similiar to the projection method,
where $p$ may only be determined up to a constant. Here, since we make use of
the Irwin crack tip solutions for some of the boundaries, the overall constant
will be constrained. For free bodies, conservation of momentum could be used
to set the constant.

\bibliography{bmg}

\end{document}
