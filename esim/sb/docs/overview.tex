\documentclass[12pt]{article}
\usepackage{amsmath,amssymb,mathpazo,fullpage,microtype}

\bibliographystyle{elsarticle-num}

\usepackage{hyperref,color}
\definecolor{webgreen}{rgb}{0,.35,0}
\definecolor{webbrown}{rgb}{.6,0,0}
\definecolor{RoyalBlue}{rgb}{0,0,0.9}

\hypersetup{
   colorlinks=true, linktocpage=true, pdfstartpage=3, pdfstartview=FitV,
   breaklinks=true, pdfpagemode=UseNone, pageanchor=true, pdfpagemode=UseOutlines,
   plainpages=false, bookmarksnumbered, bookmarksopen=true, bookmarksopenlevel=1,
   hypertexnames=true, pdfhighlight=/O,
   urlcolor=webbrown, linkcolor=RoyalBlue, citecolor=webgreen,
   pdfauthor={},
   pdftitle={Projection method problems},
   pdfkeywords={},
   pdfcreator={pdfLaTeX},
   pdfproducer={LaTeX with hyperref}
}

\renewcommand{\vec}[1]{\mathbf{#1}}
\newcommand{\p}{\partial}
\newcommand{\vu}{\vec{u}}
\newcommand{\vx}{\vec{x}}
\newcommand{\vX}{\vec{X}}
\newcommand{\vxi}{\boldsymbol\xi}

\DeclareMathOperator{\CN}{cn}
\DeclareMathOperator{\CD}{cd}
\DeclareMathOperator{\SN}{sn}
\DeclareMathOperator{\DN}{dn}
\DeclareMathOperator{\atant}{atan2}

\begin{document}
\section*{Some projection method problems}
\subsection*{Scaled coordinates for a stretching bar simulation}
Eran and his student Avraham are planning to examine simulations of a bar being
stretched using the STZ elastoplasticity model. They aim to look at a similar
problem to a previous paper~\cite{rycroft12}, but this previous work had some
severe limitations: it used made-up model parameters, and it also only employed
an explicit numerical scheme. Consequently, the elasticity constants were very
soft, and many orders of magnitude smaller than would be appropriate for
modeling a metallic glass. It is now a good time to revisit this problem. The
new projection method for elastoplasticity~\cite{rycroft15} allows us to solve
the problem in the quasi-static limit, so that realistic elastic constants can
be employed. We can look in much more detail at the interplay between the
physical model and the bar geometry.

For this problem, we work in the $xy$-plane and we consider a bar between two
pairs of walls at $x=\pm W(t)$. The bar initially covers the region $|x|<W(0)$
and $|y|<H$ where $H$ is a constant, and might have additional notches removed
from it, in order to provide a location for plastic deformation to nucleate.
Typically, the walls would move at constant velocity, so that $W(t)=W_0+Vt$ for
some constant $V$. Usually, we consider $V>0$, corresponding to stretching.
Initially, we plan to use a fixed rectangular grid for the simulation. While
this works, it is not ideal for this problem, since as the wall moves, new
layers of grid points have to be introduced. A possible rectification of this
is to work using a scaled coordinate,
\begin{equation}
X= \frac{x}{W(t)}.
\end{equation}
With respect to this coordinate, the walls remain fixed at $X=\pm 1$,
eliminating the need to continually expand the grid. One could also consider
scaling the vertical coordinate, using
\begin{equation}
Y= f(t) y
\end{equation}
for some function $f(t)$. A natural choice would be to choose $f$ so that if
the bar deformed purely elastically, then the bar height would remain fixed in
$Y$ (\textit{i.e.} taking into account the thinning due to the Poisson ratio).

This motivates an interesting question: \textbf{how can the explicit and
quasi-static simulation procedures of the recent paper~\cite{rycroft15} be best
adapted to this case?} There a number of possible considerations:
\begin{enumerate}
  \item What is the best way to represent velocity and stress? They could
    just exactly match their values in the original physical coordinate
    system, or they could be specified with respect to the scaled coordinate
    system.
  \item The recent paper~\cite{rycroft15} uses plane strain conditions. Here
    we might also be interested in plane stress conditions, which would be
    closer to an experimental test of this simulation. How does this change
    things?
  \item There are several boundary conditions of interest:
    \begin{enumerate}
      \item Pinning the bar to the wall, so that both components of velocity
	match the wall.
      \item Allowing the bar to slide up and down the wall. The horizontal
	velocity should match the wall, and the shear stress should be zero.
      \item Pulling on the bar with constant force.
    \end{enumerate}
\end{enumerate}

\subsection*{Understanding the accuracy of a fluid projection method}
One of the main outcomes of the recent paper~\cite{rycroft15} is that it
establishes a link between the projection method of Chorin for the
incompressible Navier--Stokes equations~\cite{chorin68}, and new class of
problems in elastoplasticity. It therefore provides a bridge to allow
more recent fluid simulation approaches to be translated across to
elastoplasticity.

A particularly good example is that by Yu \textit{et al.}~\cite{yu03,yu07} for
simulating an inkjet. It has some very nice properties, such as an advanced
method for handling advection, plus a finite-element (FE) projection step that
could, in principle, be much better for handling complex boundaries.
However, some of the method appears to be designed for high-Reynolds flows
where advection dominates. A good starting point would therefore be \textbf{to
examine, for a variety of different initial conditions, how different
features of the method influence its accuracy and stability.} Specifically:
\begin{enumerate}
  \item Is the intermediate MAC projection step needed? Since the final
    FE-based projection ensures that the velocity is divergence-free, it's not
    clear that this is necessary. Does it improve accuracy?
  \item There is a complicated initial procedure to calculate tangential
    derivatives. When does this offer an advantage? Can it be removed in
    some situations?
\end{enumerate}
There several ways to approach this:
\begin{enumerate}
  \item Consider a variety of different grid sizes and examine convergence
    rates of the simulation as a function of $\Delta t$ and $\Delta x$. Use
    different norms, such as $L_1$, $L_2$, and $L_\infty$. Compare the original
    simulation to the variants and see how the comparison holds up as the
    resolution is increased.
  \item Try a range of different Reynolds numbers, since certain features of the
    simulation method may only be advantageous in certain regimes.
  \item In general, the incompressible Navier--Stokes equations do not have
    closed-form solutions. However the method of manufactured solutions
    provides one way to do this, whereby a body force $\vec{f}(\vec{x},t)$ is
    introduced to induce a velocity field that has a closed form. A common
    velocity field to consider is
    \begin{equation}
      \label{eq:vel}
      \vu(\vx) = \left(
      \begin{array}{c}
	b\sin ax \cos by \sin ct \\
	-a\cos ax \sin by \sin ct
      \end{array}
      \right),
    \end{equation}
    for some constants $a$, $b$, and $c$. This is incompressible.
  \item Richardson extrapolation is another useful method for analysing
    convergence rates. It makes it possible to say something by just having a
    sequence of simulations at a finite grid sizes. Notes on this on the
    \href{http://iacs-courses.seas.harvard.edu/courses/am205/}{AM205 website}
    will be available shortly.
\end{enumerate}

\bibliography{elas}

\end{document}
